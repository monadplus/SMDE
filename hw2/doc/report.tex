\documentclass[12pt, a4paper]{article} % book, report, article, letter, slides
                                       % letterpaper/a4paper, 10pt/11pt/12pt, twocolumn/twoside/landscape/draft

%%%%%%%%%%%%%%%% PACKAGES %%%%%%%%%%%%%%%%%%%%%

\usepackage[utf8]{inputenc} % encoding

\usepackage[english]{babel} % use special characters and also translates some elements within the document.

\usepackage{amsmath}        % Math
\usepackage{amsthm}         % Math, \newtheorem, \proof, etc
\usepackage{amssymb}        % Math, extended collection
\usepackage{amsfonts}       % Math, Natural, Integers and so
\usepackage{bm}             % $\bm{D + C}$
\newtheorem{theorem}{Theorem}[section]     % \begin{theorem}\label{t:label}  \end{theorem}<Paste>
\newtheorem{corollary}{Corollary}[theorem]
\newtheorem{lemma}[theorem]{Lemma}
\theoremstyle{definition}
\newtheorem{definition}{Definition}[section]
\newenvironment{claim}[1]{\par\noindent\underline{Claim:}\space#1}{}
\newenvironment{claimproof}[1]{\par\noindent\underline{Proof:}\space#1}{\hfill $\blacksquare$}

\usepackage{hyperref}       % Hyperlinks \url{url} or \href{url}{name}

\usepackage{parskip}        % \par starts on left (not idented)

\usepackage{abstract}       % Abstract

\usepackage{tocbibind}      % Adds the bibliography to the table of contents (automatically)

\usepackage{graphicx}       % Images
\graphicspath{{./images/}}

\usepackage[vlined,ruled]{algorithm2e} % pseudo-code

% \usepackage[document]{ragged2e}  % Left-aligned (whole document)
% \begin{...} ... \end{...}   flushleft, flushright, center

%%%%%%%%%%%%%%%% CODE %%%%%%%%%%%%%%%%%%%%%

\usepackage{minted}         % Code listing
% \mint{html}|<h2>Something <b>here</b></h2>|
% \inputminted{octave}{BitXorMatrix.m}

%\begin{listing}[H]
  %\begin{minted}[xleftmargin=20pt,linenos,bgcolor=codegray]{haskell}
  %\end{minted}
  %\caption{Example of a listing.}
  %\label{lst:example} % You can reference it by \ref{lst:example}
%\end{listing}

\newcommand{\code}[1]{\texttt{#1}} % Define \code{foo.hs} environment

%%%%%%%%%%%%%%%% COLOURS %%%%%%%%%%%%%%%%%%%%%

\usepackage{xcolor}         % Colours \definecolor, \color{codegray}
\definecolor{codegray}{rgb}{0.9, 0.9, 0.9}
% \color{codegray} ... ...
% \textcolor{red}{easily}

%%%%%%%%%%%%%%%% CONFIG %%%%%%%%%%%%%%%%%%%%%

\renewcommand{\absnamepos}{flushleft}
\setlength{\absleftindent}{0pt}
\setlength{\absrightindent}{0pt}

%%%%%%%%%%%%%%%% GLOSSARIES & ACRONYMS %%%%%%%%%%%%%%%%%%%%%

%\usepackage{glossaries}

%\makeglossaries % before entries

%\newglossaryentry{latex}{
    %name=latex,
    %description={Is a mark up language specially suited
    %for scientific documents}
%}

% Referene to a glossary \gls{latex}
% Print glossaries \printglossaries

\usepackage[acronym]{glossaries} %

% \acrshort{name}
% \acrfull{name}
% \newacronym{kcol}{$k$-COL}{$k$-coloring problem}

%%%%%%%%%%%%%%% COMMANDS

\newcommand{\Z}{\mathbb{Z}}

%%%%%%%%%%%%%%%% HEADER %%%%%%%%%%%%%%%%%%%%%

\usepackage{fancyhdr}
\pagestyle{fancy}
\fancyhf{}
\rhead{Arnau Abella \- MIRI}
\lhead{Statistical Modelling and Design of Experiments}
\rfoot{Page \thepage}

%%%%%%%%%%%%%%%% TITLE %%%%%%%%%%%%%%%%%%%%%

\title{%
  SMDE: Homework 2
}
\author{%
  Arnau Abella \\
  \large{Universitat Polit\`ecnica de Catalunya}
}
\date{\today}

%%%%%%%%%%%%%%%% DOCUMENT %%%%%%%%%%%%%%%%%%%%%

\begin{document}

\maketitle

%%%%%%%%%% Exercise 1

\section{Simulate your data}

The first exercise can be found on the directory \code{ex1/}.

The generated dataset can be found at \code{ex1/dataset.csv}.

This dataset was generated following the instructions of the exercise. The choosen answer variable is:

\begin{equation}
  Answer = f_1 + f_2 + f_4 + 5 f_5
\end{equation}

In order to generate a similar dataset, you only have to run the haskell script \code{./ex1/Main.hs} that I prepared.

Before that, you only need to install \href{https://nixos.org/download.html}{Nix} which will bring all the dependencies to your environment to be able to run the script.

%%%%%%%%%%%%% Exercise 2

\section{Obtain an expression to generate new data}

For exploring the dataset relationships of the different factors I used two techniques:

\begin{itemize}
  \item Multiple Linear Regression Model.
  \item Principal Component Analysis.
\end{itemize}

\newpage

The resulting expression from the MLRM, which is the one I decided to use for the following sections, is

\begin{equation}
  Answer' = 0.005562 + 1.032f_1 + 0.988f_2 + 0.988f_4 + 4.94f_5
\end{equation}

which is a very good approximation of the real answer.

There is a dedicated document to this part that can be found at \code{ex2/ex2.pdf}. All material used for that part can also be found at \code{ex2/}

\section{Simulate New Data}

For this part, I used the expression obtained from the \textit{LRM} in the previous section and the framework \textit{GPSS} to simulate the data. All code can be found at \code{ex3/}.

\subsection{Validation of the simulation}

In order to validate we used some operational validation techniques:

\begin{itemize}
  \item Black Box validation

    We compared the real system data with the simulated data to validate its accuracy. We used \textit{ANOVA} test to validate that they produce equivalent data.

  \item GPSS Traces

    We used the simulation traces to compare the results, and analyze if the logic of the events are coherent with the understanding the experts have of the system.
\end{itemize}

\subsection{Design of Experiment}

In this section we used the formula from the \textit{LRM} to generate thousands of new records and analyze the interaction of these records with the answer dependent variable. The code for this section can be found at \code{ex4/}.

For this part, I wrote another haskell script \code{./ex4/Main.hs} to run the Yates algorithm on the output of the generated data. The result of Yates can be found at \code{ex4/yates.txt}. The file has the following structure:

\begin{listing}[H]
  \begin{minted}[xleftmargin=10pt,linenos,bgcolor=codegray]{bash}
[(+),(+),(+),(-),(-),(-),(+),(+),(+),(+)]  =  7.21245
[(+),(+),(+),(-),(-),(+),(-),(-),(-),(-)]  =  1.50318
[(+),(+),(+),(-),(-),(+),(-),(-),(-),(+)]  =  2.503016
[(+),(+),(+),(-),(-),(+),(-),(-),(+),(-)]  =  2.753091
[(+),(+),(+),(-),(-),(+),(-),(-),(+),(+)]  =  2.633003
[(+),(+),(+),(-),(-),(+),(-),(+),(-),(-)]  =  -5.92082
[(+),(+),(+),(-),(-),(+),(-),(+),(-),(+)]  =  1.74188
[(+),(+),(+),(-),(-),(+),(-),(+),(+),(-)]  =  1.9131
[(+),(+),(+),(-),(-),(+),(-),(+),(+),(+)]  =  4.18690
[(+),(+),(+),(-),(-),(+),(+),(-),(-),(-)]  =  -1.41055
[(+),(+),(+),(-),(-),(+),(+),(-),(-),(+)]  =  -2.581421
[(+),(+),(+),(-),(-),(+),(+),(-),(+),(-)]  =  1.66423
[(+),(+),(+),(-),(-),(+),(+),(-),(+),(+)]  =  2.787
[(+),(+),(+),(-),(-),(+),(+),(+),(-),(-)]  =  -3.885857
[(+),(+),(+),(-),(-),(+),(+),(+),(-),(+)]  =  6.29247
[(+),(+),(+),(-),(-),(+),(+),(+),(+),(-)]  =  -2.425661
[(+),(+),(+),(-),(-),(+),(+),(+),(+),(+)]  =  -1.2188
[(+),(+),(+),(-),(+),(-),(-),(-),(-),(-)]  =  1.074661
[(+),(+),(+),(-),(+),(-),(-),(-),(-),(+)]  =  -1.301479
[(+),(+),(+),(-),(+),(-),(-),(-),(+),(-)]  =  1.223755
[(+),(+),(+),(-),(+),(-),(-),(-),(+),(+)]  =  -5.17829
  \end{minted}
  \caption{\textit{head} of \code{ex4/yates.txt}}
\end{listing}

The file contains 1024 so it is time consuming to analyze by hand. I made a small script to get the most significant factors. As expected, the factors are consistent with the experiment results.

\end{document}
